\chapter{The register organisation}
\label{cha:RegisterOrganisation} 

\par \noindent The LUMEN interpreter provides two register banks directly accessible by the programmer. One register bank is reserved for the 8-bit 
instructions and the other register, the 32-bit register is used for the floating point operations. Each register bank contains 16 registers.
15 registers are general purpose and one register is the accumulator register. \bigskip

\par \noindent The program counter register points to the next instruction to be executed in the program memory. The stack pointer register points to the current
stack position. 

\subsection {8-bit register bank}
\par \noindent The 8-bit register bank consist out of 16 8-bit registers, R0, R1, \ldots , R14 and a special HL register. The HL register, also known as 
the accumulator register contains results given by several instructions and should not be used to store values. Only 8-bit unsigned integers can be 
stored in these registers. 

\subsection{32-bit register bank}
\par \noindent The 32-bit register bank consist out of 16 32-bit registers, R0, R1, \ldots , R14 and a special HL register. The HL register, also known as 
the accumulator register contains results given by several instructions and should not be used to store values. Only 32-bit  single precision floats 
can be stored in these registers. Pushing values of the register to the stack is only possible with a special instruction.

\subsection {Program counter register}
\par \noindent The program counter is a 16-bit register. On power up the program counter is initialized to address 0x0000 and the instruction find
 on this location in ROM is executed. From this point on the program counter is controlled indirectly by the program instructions themselves 
 that were generated by the programmer. 

\subsection{Stack pointer register}
\par \noindent The stack pointer register is a 8-bit register and used to keep track of the top of the stack. The stack is used for saving variables, returning 
addresses, passing arguments to subroutines and various other uses that might be conceived by the programmer. The instructions PUSH and CALL put 
information onto the stack. The instruction POP takes information off the stack. \bigskip

\par \noindent As information is put onto the stack, the stack grows downwards in RAM memory. As a result, the stack pointer should always be pointing at the highest
location of RAM space that has been allocated to use by the stack. On power-up the stack pointer is initialized to 0x00.